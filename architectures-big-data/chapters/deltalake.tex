\section{Delta Lake}
In un contesto big data esiste la sfida di rendere transazionali le operazioni.
In particolare potrebbe essere necessario garantire consistenza e comunicazione tra
i processi che stanno svolgendo particolari operazioni.

\subsection{Transazionalità}
Per garantire transazionalità è necessario un controllo sulla concorrenza.

\subsubsection{Transazione}
Sequenza di operazioni che soddifano le proprietà ACID.

\subsubsection{Rollback}
Riportare il sistema allo stato precendete all'operazione eseguita.

\subsubsection{Proprietà ACID}
Le proprietà che sono necessarie per garantire transazionalità sono:
\begin{itemize}
    \item \emph{Atomicity}: ogni operazione di una transazione è atomica
    \item \emph{Consistentcy}: una transazione deve mantenere la consistenza del sistema
    \item \emph{Isolation}: esecuzioni concorrenti portano allo stesso risultato di esecuzioni sequenziali
    \item \emph{Durability}: una transazione non può essere ripristinata dopo un commit
\end{itemize}

\subsection{Transaction log e shipping}
Per garantire consistenza e replicazione si possono applicare \emph{transaction log} e
\emph{transaction shipping}.

\subsubsection{Transaction log} 
Storia delle operazioni eseguite da un database, possono essere utili per effettuare
rollback in casi di errore di alcune transazioni che hanno lasciato
il sistema in uno stato incosistente. Questi file possono diventare molto grandi.

\subsubsection{Transaction shipping}
I log possono essere mandati a un server per replicare lo stato del database e avere
una copia esatta del sistema. È un'operazione molto pesante poichè bisogna scrivere
tutte le operazioni che sono effettuate sulle tabelle.

\subsection{Delta Lake}
Delta lake ha l'obiettivo di garantire le ACID properties su HDFS, quindi su 
storage distribuiti.

\subsubsection{Apache Parquet}
Parquet è un formato con cui mantenere le tabelle. Usa memorizzazione 
per colonna piuttosto che per riga, per velocizzare le operazioni di select.\\
Dividendo per colonna poi si può ottimizzare lo spazio richiesto per la codifica
dei valori nella tabella. Infatti, i valori di una colonna sono più omogenei
di quelli presenti in un'intera riga, quindi richiedono meno spazio e si possono
comprimere. Parquet non è abbastanza per garantire transazionalità.

\subsubsection{Write-ahead log} 
Si mantengono quali oggetti fanno parte di una delta table, oltre ad 
aclune infomazioni statistiche quali massimo, minimo, count, etc \dots\\
Si scrivono poi le operazioni da fare sulle tabelle nel log, prima di eseguirle, 
oltre a quelle da applicare in caso di rollback.

\subsubsection{Funzionalità}
Le funzionalità principali di un Delta Lake, che lo rendono propenso a un forte lock-in, sono:
\begin{itemize}
    \item \emph{Time travel}: esiste la possibilità di riportare lo stato dello storage a un determinato
    snapshot temporale.
    \item \emph{UPSERT, DELETE e MERGE}: riscrivono in modo efficiente gli oggetti per 
    applicare le operazioni richieste
    \item \emph{Streaming efficiente}: le informazioni sono scritte in modo veloce poichè 
    si suddividono in piccoli blocchi. I blocchi si uniscono dopo per velocizzare le letture
    \item \emph{Caching}: i nodi del cluster possono cachare informazioni
    \item \emph{Schema evolution}: si può cambiare lo schema delle tabelle continuando 
    a leggere i vecchi file Parquet 
\end{itemize}

