\section{Graphs}

\subsection{Ways to generate a random graph}

What we want to do is to generate a random graph with a given density.
Formally, we want a $G(V,A)$ with density: 
$$\rho = \frac{r}{n(n-1)}$$
Thus the number of required edges are 
$$p = \rho m$$

\paragraph{Edges coding}
We must image a coding between natural numbers and edges, as we need it 
for the following algorithms, some examples are: 
\begin{itemize}
    \item \dots
\end{itemize}

\paragraph{First algorithm}
The first idea is to select at random $r$ edges from 
the set of all possible $m$ ones. 

The time complexity is $O(r)$ and the space complexity is $O(r)$.

\paragraph{Second algorithm}
A second algorithm could be selecting $r$ edges at random, maintaining 
a set or vector with boolean values to avoid repetitions. 

The space complexity is $O(m)$ with a boolean vector, or $O(r)$ with a set. 
We could use a bitvector to reduce memory, while the time complexity requires 
a more detailed analysis.

Let $W_k$ be the random variable that indicates how many conflicts we 
get selecting at random from $m$ with $k$ already selected edges.

\dots

$E[W] = p$

Time complexity now now depends on the structure we use to store the already seen 
edges. 
The worst case complexity is infinite. 

\paragraph{Third algorithm}
We can allocate an array with $m$ elements, select a random $i$ and swap it 
with the first element. 
Proceed with selecting a random number between in $[2,m]$ and swap it with the second 
element. So on until we have $p$ edges.

The time complexity is $O(p)$ and so is the space complexity is $O(m)$.

\paragraph{Fourth algorithm}
An improvement can be made in the space of the third algorithm, we can store 
at most $O(p)$ elements taking care of the swaps. An extraction then search in the 
$p$ auxiliary elements, and if found it uses the swap indicated there, if not, it 
uses the extracted random number as edge.

Time complexity now depends on the way we store the additional $p$ elements. 
If we store them in a linked list, time complexity becomes $O(p^2)$ and space $O(p)$.

