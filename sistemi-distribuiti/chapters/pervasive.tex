\section{Sistemi pervasivi}

\subsection{Sensori}

\paragraph{Transducer}
Apparati che trasformano una forma di energia in un'altra.

\paragraph{Funzionamento sensori}
Nella pratica si utilizza un Transducer per catturare
un fenomeno fisico, trasformandolo in un segnale elettronico.
Uno dei parametri di tale osservazione è la frequenza con 
cui si osserva il fenomeno fisico.

\paragraph{Tipi di sensori}
Si possono disinguere tra i sensori fisici tre tipologie:
\begin{itemize}
    \item \emph{Movimento}: misurano forse di accellerazione 
    e rotazione sui tre assi
    \item \emph{Ambientali}: misurano parametri ambientali, 
    quali la temperatura, pressione, illuminazione e umidità
    \item \emph{Posizione}: misurano la posizione fisica del 
    dispositivo
\end{itemize}

\paragraph{Componenti si un sensore}
Tra i componenti di un sensore si individuano sicuramente:
\begin{itemize}
    \item \emph{Sensing subsystem}: parte che cattura i fenomeni
    fisici
    \item \emph{Processing subsystem}: processing dei segnali individuati
    \item \emph{Wireless communication}: comunicazione con altri apparati
    \item \emph{Power source}: fonte di energia, batteria, pannello solare, etc.
\end{itemize}

\paragraph{Sensori virtuali}
Un esempio di un sensore virtuale sono le Google Places API, 
uno use-case potrebbe essere individuare la posizione 
tramite un sensore classico, per poi utilizzare 
un servizio, come ad esempio quella API, per capire cosa c'è 
nelle vicinanze oppure che tempo fa. 

Si può pensare a quei servizi come una sorta di sensori virtuali.

\paragraph{Attuatori}
Sono particolari tipi di transducer, sono alimentati e convertono 
energia in azione, effettuano quindi movimenti o operazioni 
di switch.

Un esempio sono dispositivi che aprono porte, aprono valvole, 
etc.

\paragraph{Discovery e pairing}
Smart devices possono apparire e scomparire frequentemente in rete. 
Nella pratica serve un modo di registrare un dispositivo, assegnandogli un 
indirizzo, simile a DHCP, per poi associare il device.

\subsection{Acquisizione di dati}

\paragraph{Desiderata}
I problemi di interrogazione dati in un sistema pervasivo differisce 
dai classici problemi legati ai database.

Esiste la necessità di interrogare sensori in modo continuo, l'opposto 
delle query in database, che sono solitamente one-time, 
i dati hanno inoltre una forte caratterizzazione spazio-temporale, ad esempio, 
tenedo sotto controllo una temperatura, mi basterebbe capire quando 
questa cambia, senza continuare ad interrogare.

Si fa notare che non è ragionevole ripetere una query one-time ogni secondo 
per simulare un flusso continuo di dati.

\paragraph{Processing}
Esistono alcuni approcci semplici per il processing dei dati dai sensori:
\begin{itemize}
    \item \emph{Batch processing}: i sensori possono fare buffering, e poi processati offline sulla base station, 
    o su un server.
    Non c'è processing sul sensore
    \item \emph{Sampling}: si campionano le misurazioni, anche se non 
    è una misurazione completa come nel primo approccio, esistono delle garanzie 
    di qualità
    \item \emph{Sliding window}: si fornisce una risposta approssimata 
    basata su un gruppo di letture consecutive, la base station 
    riceve questa misurazione aggregata approssimata.
\end{itemize}

\paragraph{In-network query processing}
L'idea di base di questo approccio è quello 
di creare una overlay network, ad esempio uno spanning tree, 
per aggregare i dati in modo gerarchico.

\paragraph{Approcci model-based}
I dati raccolti possono avere forte correlazione spazio-temporale.
Si potrebbe sfruttare tale correlazione per ridurre il sampling necessario.

Tale correlazione è utile sia per ridurre il numero di campionamenti 
necessari, sia per pulire i dati provenienti dai sensori, tipicamente 
poco precisi, sfruttando tecniche statistiche.