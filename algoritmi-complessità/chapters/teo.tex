\section{Teoria della complessità di approssimazione}

In questa sezione si affrontano alcune tematiche, in particolare il teorema PCP, 
con il quale di riuscirà a dimostrare l'inapprossimabilità di MaxEkSat e Indipendent set.

\subsection{Verificatori}
Il concetto di verificatore assume diverse sfaccettature nel corso 
delle sezioni che seguono, ma sostanzialmente è una macchina che si occupa di verificare 
la decisione di un certo problema per un certo input.

\subsubsection{NP attravero testimoni}
Una Macchina di Turing verificatore per un problema di decisione $L \subseteq 2^*$, riceve due input, $x$ e $w$, testimone, 
e si comporta come segue:
\begin{enumerate}
    \item se $x \notin L$ risponde no qualunque sia $w$
    \item se $x \in L$, allora $\exists w\in 2^*$ tale che la macchina risponde sì
\end{enumerate}
Devono valere però:
\begin{enumerate}
    \item $|w| \leq P(|x|)$ ovvero la lunghezza di $w$ è polinomiale nell'input
    \item la macchina lavora in tempo polinomiale
\end{enumerate}

\begin{remark}
    Si può pensare alla classe NP come una classe di problemi facilmente verificabili, ma difficilmente 
    risolvibili.
\end{remark}
\begin{remark}
    Esistono problemi nè risolvibili nè  verificabili facilmente.
\end{remark}
\begin{theorem}
    Un problema $L \subseteq 2^* \in $NP se e solo se esiste una Macchina di Turing deterministica $V$, verificatore,
    e un polinomio $P()$, tali che:
    \begin{enumerate}
        \item $V(x,w)$ lavora in tempo polinomiale in $|x|$
        \item $\forall x \in 2^*$:
        \begin{enumerate}
            \item se $x \in L$, $\exists w \in 2^*$ tale che $|w| \leq P(|x|)$ e $V(x,w) = \mathit{yes}$
            \item se $x \notin L$, $\forall w \in 2^*$ $|w| \leq P(|x|)$, $V(x,w) = \mathit{no}$
        \end{enumerate}
    \end{enumerate}
\end{theorem}

\subsubsection{NP con oracolo}
L'idea di una Macchina di Turing con oracolo presuppone l'esistenza, oltre 
al nastro classico, di un nastro dell'oracolo.

La Macchina può entrare in uno stato di query all'oracolo, 
in cui lo interroga e l'oracolo risponderà, facendole cambiare stato di conseguenza.

\begin{theorem}
    Un problmema $L \subseteq 2^* \in $ NP se esiste una Macchina di Turing con oracolo $V$ e
    un polinomip $P()$, tale che:
    \begin{enumerate}
        \item $V(x)$ lavora in tempo polinomiale in $|x|$
        \item $V(x)$ effettua al più $P(|x|)$ query all'oracolo
        \item $\forall x \in 2^*$:
        \begin{enumerate}
            \item se $x \notin L$, $V(x) = no$ qualunque sia l'oracolo
            \item se $x \in L$, $V(x) = yes$ per una qualche stringa dell'oracolo
        \end{enumerate}
    \end{enumerate}
\end{theorem}

\subsubsection{Verificatore probabilistico}
In questo caso si estende il concetto di verificatore con 
oracolo aggiungendo un nastro di numero casuali che la Macchina può leggere.

Date due funzioni 
$$q,r : \mathbb{N} \longrightarrow \mathbb{N}$$
$PCP[r,q]$ è la classe di linguaggi $L \subseteq 2^*$ per i quali esiste un verificatore probabilistico
con le seguenti proprietà: 
\begin{enumerate}
    \item $V(x)$ lavora in tempo polinomiale
    \item $V(x)$ fa al massimo $q(|x|)$ query all'oracolo
    \item $V(x)$ usa al più $r(|x|)$ bit casuali
    \item $\forall x \in 2^*$:
    \begin{enumerate}
        \item Se $x \notin L$, $V(x) = no$ con probabilità $\geq \frac{1}{2}$
        \item Se $x \in L$, $V(x) = yes$ con probabilità $1$ per almeno
        un $w \in 2^*$
    \end{enumerate}
\end{enumerate}

\begin{remark}
    Vale che:
    \begin{itemize}
        \item $NP = PCP[0,\mathit{Poly}] = \cup_{p : \mathit{polinomio}}PCP[0, p]$ 
        \item $P = PCP[0,0]$
    \end{itemize}
\end{remark}

\subsubsection{Teorema PCP}
Definiti i verificatori probabilistici, si passa ora al teorema PCP, 
che in un certo senso dice che il non determinismo può essere portato ad una costante, 
a patto che si facciano una quantità logaritmica di scelte casuali.

\begin{theorem}
    $PCP[O(\log n), O(1)] = NP$
\end{theorem}

% \subsubsection{Verificatore NP in forma canonica}

% \subsection{Problemi inapprossimabili}

% \subsubsection{Inapprossimabilità MaxEkSat}

% \subsubsection{Inapprossimabilità Max Indipendent Set}
