\subsection{Tecniche di pricing}
L'idea è quella di attribuire ad ogni elemento da inserire in 
una soluzione un costo. I costi permettono di scegliere gli elementi 
più vantaggiosi e di analizzare il tasso di approssimazione di questi algoritmi.

\subsubsection{Minimum Set Cover}
Nel problema di Set Cover l'obiettivo è quello di coprire l'universo $U$, 
utilizzando seubset di esso che hanno un costo. Si vuole ottenere il costo
minimo, dato come somma dei costi dei set che si sono scelti.

Formalmente: \\
\emph{Input}: $s_1, \dots, s_m$, $\bigcup_{i=1}^m s_i = U$, $|U| = n$\\
\emph{Output}: $C = \{s_1, \dots, s_n\}$, tali che, $\bigcup_{s_i \in C} s_i = U$\\
\emph{Costo}: $w = \sum_{s_i \in C} w_i$\\
\emph{Tipo}: min\\

\paragraph{Funzione armonica}
La funzione armonica è definita come: 
\begin{equation}
    \begin{aligned}
        H: \mathbb{N}^{>0} \rightarrow \mathbb{R}\\
        H(n) = \sum_{i = 1}^{n} \frac{1}{i}  
    \end{aligned}
\end{equation}
Vale la seguente proprietà: 
% \begin{equation}
%     \begin{aligned}
%         H(n) \leq 1 + \int_{1}^{n} \frac{1}{x} \,dx \leq 1 +
%         \big[ \ln n\big]_1^n = 1 + \ln n\\
%         \ln (n+1) \leq H(n) \leq 1 +\ln n
%     \end{aligned}
% \end{equation}
$$\ln (n+1) \leq H(n) \leq 1 +\ln n$$

\paragraph{Greedy Set Cover}
L'algoritmo effettua ad ogni iterazione una scelta greedy, 
si sceglie l'insieme che minimizza il rapporto tra prezzo e copertura 
dell'universo.

\begin{algorithm}[H]
    \SetAlgoLined
    \KwIn{$s_1, \dots, s_m$, $w_0, \dots, w_n$}
    \KwResult{Scleta di sottoinsiemi che copre l'universo}
     $R \gets U$\\
     $C \gets \emptyset$\\
     \While{$R \neq \emptyset$}{
         $S_i \gets \min(s_1, \dots, s_m, \frac{w_i}{|S\cap R|})$\\
         $C.add(S_i)$\\
         $R \gets R \setminus S_i$

     }
     \Return{$C$}
     \caption{GreedySetCover}
\end{algorithm}

\begin{remark}
    \label{oss1set}
    Il costo della soluzione equivale a $$w = \sum_{s\in U}c_s$$ ovvero la somma dei costi 
    degli insiemi scelti.
\end{remark}
\begin{remark}
    \label{oss2set}
    Per ogni $k$, il costo degli elementi in $s_k$, ottengo 
    $$\sum_{s \in S_k}C_s \leq H(|S_k|) \cdot  W_k$$
\end{remark}
\begin{proof}
    Sia $S_k = \{s_1, \dots, s_d\}$ un insieme tra quelli da scegliere, e siano 
    i suoi elementi elencati in ordine di copertura\footnote{Per chiarezza, l'insieme $S_k$ non verrà scelto, 
    ma i suoi elementi saranno coperti da altri insiemi che intersecano con esso.}.

    Consideriamo ora l'istante in cui si copre $S_h$ tramite un quale insieme $S_h$.
    Si può notare che, visto che gli elementi sono in ordine di copertura: 
    $$R \supseteq \{S_j, \dots, S_d \}$$
    Inoltre, visto che gli elementi di $S_k$ sono in ordine di copertura: 
    $$|S_k \cap R| \geq d-j+1$$
    Riguardo al costo dell'elemento $j$, e in geenerale per tutti i $j$, vale:
    \begin{equation}
        \begin{aligned}
            C_{s_j} = \frac{W_h}{|S_h \cap R|} \leq \frac{W_k}{|S_k \cap R|} && \text{\emph{h} minimizza quel rapporto}\\
            \leq \frac{W_k}{d-j+1} && \text{equazione precedente}\\
        \end{aligned}
    \end{equation}
    Considerando ora tutti gli elementi di $S_k$: 
    \begin{equation}
        \begin{aligned}
            \sum_{s\in S_k} C_s \leq \sum_{j=1}^{d}\frac{W_k}{d-j+1} = \frac{W_k}{d} + \frac{W_k}{d-1} \dots && \text{La relazione vale per tutti i j}\\
            = W_k(1 + \frac{1}{2}, \dots, \frac{1}{d}) = H(d)W_k = H(|S_k|)W_k && \text{Sviluppo e raccolgo, ottengo l'oss.}
        \end{aligned}
    \end{equation}
\end{proof}
\begin{theorem}
    Greedy Set Cover fornisce una $H(M)$-approssimazione per Set Cover, dove $M=\max_i|S_i|$
\end{theorem}
\begin{proof}
    Sia il peso della soluzione ottima $$w^* = \sum_{S_i \in C^*} w_i$$
    Per l'osservazione \ref{oss2set} vale che: 
    $$w_i \geq \frac{\sum_{s \in S_i}C_s}{H(|S_i|)} \geq \frac{\sum_{s \in S_i}C_s}{H(M)}$$
    Visto che gli $s_i \in C^*$ sono una copertura, per l'osservazione \ref{oss1set}:
    $$\sum_{S_i \in C^*}\sum_{s \in S_i}C_s \geq \sum_{s \in U} C_s = w$$
    Inoltre, vale che, sfruttando le due disequazioni appena scritte: 
    \begin{equation}
        \begin{aligned}
            w^* = \sum_{S_i \in C^*} w_i \geq \sum_{S_i \in C^*}\frac{\sum_{s \in S_i}C_s}{H(M)} \geq \frac{w}{H(M)}\\
            \implies \frac{w}{w^*} = H(M)
        \end{aligned}
    \end{equation}
\end{proof}
\begin{corollary}
    Greedy Set Cover fornisce una $O(\log n)$-approssimazione, non quindi costante 
    come gli algoritmi precedenti.
\end{corollary}
\begin{remark}
    L'analisi è tight, non esiste un algoritmo migliore, quindi
    Greedy Set Cover $\notin$ APX, bensì, $\in \log(n)$-APX, una classe in cui 
    si accetta un'approssimazione che peggiora logaritmicamente nell'input.
    Esistono varie $f$-APX.
\end{remark}
\begin{proof}
    Per dimostrare la tightness, ecco un esempio in cui Greedy Set Cover va male.

    Consideriamo l'insieme $S$ di set tra cui scegliere così formato:
    \begin{enumerate}
        \item Due insiemi grandi $\frac{n}{2}$ che uniti coprono tutti gli elementi, 
        di costo $1+\epsilon$
        \item Un insieme che copre $\frac{n}{4}$ elementi degli insiemi 1 e 2 al punto 1, di costo 1
        \item Un insieme che copre $\frac{n}{8}$ elementi degli insiemi 1 e 2 al punto 1, di costo 1
        \item Un insieme che copre $\frac{n}{16}$ elementi degli insiemi 1 e 2 al punto 1, di costo 1\\
        \dots
    \end{enumerate}
    Al primo passo, si sceglie l'insieme del punto 2, visto che
    il suo costo equivale a $\frac{1}{\frac{n}{2}} = \frac{2}{n}$, mentre il costo 
    di entrambi gli insimemi al punto 1, $\frac{1+\epsilon}{\frac{n}{2}} = \frac{2+2\epsilon}{n}$

    Al secondo passo, si preferisce ai primi due l'insieme al punto 3, il calcolo è simile al punto precedente.\\
    \dots
    
    Il costo che si ottiene è $w = \log n$
    La soluzione ottima sarebbe quella di prendere al passo 1 i primi due insiemi, in modo da coprire l'intero 
    universo, ovvero $w^* = 2 + 2\epsilon$.

\end{proof}